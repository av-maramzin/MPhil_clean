\chapter{Related Work}
\label{related_work}
\quad 


\section{Automatic parallelization in compilers}
\label{related_work_autopar}
\quad There is a large body of work on automatic and semi-automatic parallelization of sequential legacy code \cite{6813266}, \cite{article12345}, \cite{Bacon:1994:CTH:197405.197406}, \cite{Kennedy:2001:OCM:502981}. Years have been spent building various compiler infrastructures for research on parallelizing \cite{Kennedy:2001:OCM:502981} and optimizing \cite{Muchnick:1998:ACD:286076} compilers. Research on automatic parallelization has produced several high-performance parallelizing compilers yielding high-performance parallel code, e.g., SUIF (Stanford University Intermediate Format) compiler \cite{546613},\cite{10.5555/891422},\cite{suif_compiler}, Polaris \cite{polaris}, \cite{10.1109/M-PDT.1994.329796} and LLVM (Low-Level Virtual Machine) \cite{llvm-compiler-infrastructure},\cite{Lattner:2004:LCF:977395.977673}. Industry has its own well-known and well-established parallelizing compilers like Intel C/C++ Compiler (ICC) \cite{icc-compiler} or GNU Compiler Collection (GCC) project.\newline\null
\quad It is well understood how to parallelize scientific C and Fortran code with perfectly nested loops that operate over flat arrays \cite{97902}, or even non-perfectly nested ones \cite{10.1145/263699.263719}. Furthermore, there are works on how to parallelize sequential loops across procedure boundaries \cite{10.1145/125826.126055},\cite{10.5555/645671.665383}, or with some prior enabling transformations like array privatization \cite{10.1145/158511.158515}. These works, however, focus on loops over arrays. Some research efforts deal with larger code structures
\cite{1299188}. Decoupled software pipelining is a compilation technique to recognize pipelines and to distribute the pipeline stages across threads \cite{1540952}, \cite{10.1145/1400112.1400113}.\newline\null
\quad Despite all the research efforts automatic parallelization still remains largely limited to sequential scientific codes, where DOALL type loops operate over flat arrays with boundaries known at compile-time. There is a range of reasons that complicate the applicability of automatic parallelization techniques to a wider areas. Dependence analysis is hard for code that uses indirect addressing, pointers, recursion, or indirect function calls because it is difficult to detect such dependencies at compile time. Loops often have an unknown number of iterations. Accesses to global resources are difficult to coordinate in terms of memory allocation, I/O, and shared variables. Irregular algorithms that use input-dependent indirection interfere with compile-time analysis and optimization. All enumerated complications are very typical to the real world code. As we demonstrate in Section \ref{backgrnd_challenges_automatic} all above problems can even materialize on a relatively simple and highly parallel suite of NASA Parallel Benchmarks (NPB) \cite{nasa-parallel-benchmarks}.\newline\null
\quad There is an alternative research direction. Thread-level speculation systems (TLS) can avoid the problem of program analysis by speculating the non-existence of dependencies [13]. 

This requires, however, a large amount of hardware support for handling thread contexts, but also extensions to the memory system to detect dependence violations [13,22,35] and initiate recovery process. Thus, TLS has a high hardware cost which should be
considered carefully. 

Overall, speculative parallelization targets the inner program loops [1,3,22,24,26] and cannot handle coarse-grain parallelism, especially not if the loops contain I/O or other system calls. 

structures. The dependencies are hard to analyze statically and this seriously limits the applicability of those techniques.

In contrast, analysis of parallel code and discovery of parallel
patterns are relatively unexplored areas.

To exploit the potential of upcoming multi-cores, it will be necessary to extend the scope of automatic parallelization to a broader class of programs containing complex pointer-based code and finding coarse-grain parallelism. Compilers cannot make such decisions in general as they cannot infer the high level semantics of the program.

\section{Machine learning in compilers}
\label{related_work_ml}
\quad Correctness is the most important property of the code. Although important, the running time or any other type of code performance characteristic is not always vitally critical. As all machine learning based techniques have always been characterized by their inherent and ineradicable errors \cite{James:2013:ISL:2517747}, the field of compilers has never been the primary target for these methods. Nonetheless, these techniques have found their application to some problems within the field.\newline\null
\textit{ML in Compiler Optimization.}
Usually, machine learning based methods target problems, where a misprediction will only lead to a hampered performance and not a functional failure. For example, machine learning based methods have been used for finding the most optimal compiler optimization parameters like predicting the optimal loop unroll factor \cite{4907653,1402082} or determining whether or not a function should be inlined \cite{Zhao2003ToIO,1559966}. These works are supervised classification problems and they target a fixed set of compiler options, by representing the optimization problem as a multi-class classification problem where each compiler option is a class. Recent works try to do scheduling and optimization of parallel programs for heterogeneous multi-cores. For example, Hayashi \emph{et al.}~\cite{Hayashi:2015:MPH:2807426.2807429} extracts various program features during compilation for use in a supervised learning prediction model aiming at the optimality of CPU vs. GPU selection. Evolutionary algorithms like generic search are often used to explore a large design space. Prior works~\cite{Almagor:2004:FEC:997163.997196,Cooper:2005:AAC:1065910.1065921,Ashouri:2017:MMC:3132652.3124452} have used evolutionary algorithms to solve the phase ordering problem (i.e.\ in which order a set of compiler transformations should be applied).\newline\null
\textit{Machine Learning and Parallelization.}
The application of machine learning based methods to the problem of software parallelization has not yet found a widespread practical utility. Mispredictions regarding the code parallelizability can lead to a broken dependency property and thus incorrect program execution. Nonetheless, there have already been works on predicting loop parallelizability, like the approach of Fried \emph{et al.}~\cite{fried_ea:2013:icmla}. Fried \emph{et al.} train a supervised learning algorithm on code hand-annotated with OpenMP parallelization directives to create a loop parallelizability predictor. These directives approximate the parallelization that might be produced by a human expert. Fried \emph{et al.}~focus on the comparative performance of different ML algorithms and studies the predictive performance that can be achieved on the problem and does not produce any practical application.\newline\null
\section{Discovery: data structure recognition}
\label{related_work_dcp}
\quad The idea of automatic discovery of higher-level entities in programs is not a new one. This discovery problem is closely interlinked and entangled with alias analysis techniques \cite{Muchnick:1998:ACD:286076} like points-to analysis \cite{Emami:1994:CIP:178243.178264}. The Points-to analysis is a variation of data flow analysis techniques. The final output is the sets of pairs of the form (\textit{p}, \textit{x}) (pointer variable \textit{p} points to a stack-allocated variable \textit{x}). These techniques are aimed at getting aliasing information regarding stack-allocated pointers.\newline\null
\quad The problem of understanding heap-directed pointers and heap-allocated linked data structures these pointers might point to is addressed with a family of static analysis techniques collectively known as shape analysis. Shape analysis techniques can be used to verify properties of dynamically allocated data structures in compile time. These are among the oldest and most well-known techniques. Three-valued logic \cite{Sagiv:1999:PSA:292540.292552}\cite{Wilhelm:2000:SA:647476.760384} can be used as an example. The technique proposes the construction of a mathematical model consisting of logical predicate expressions. The latter correspond to certain pointer operating imperative language program statements. An abstract interpretation of these statements leads to the construction of sets of shape graphs at various program points. Shape graphs approximate the possible states of heap-allocated linked data structures and answer the questions such as node reachability, data structure disjointness, cyclicity, etc. The major limitation of these simplified mathematical models is the lack of precision high level of abstraction leads to. The problem of precise shape analysis is provably undecidable.\newline\null
\quad The work of \cite{Ghiya:1996:TDC:237721.237724} proposes a simplified and hence more practical implementation of shape analysis. Authors propose to use direction \textit{D} and interference \textit{I} matrices instead of complex mathematical models to derive shape information on heap-allocated data structures. The entry of direction matrix \textit{D[p,q]} says if there exists a path from a node referred to by \textit{p} to a node referred to by q. In other words, if we can enter a path within the data structure through \textit{p} and exit through \textit{q}. The entry of interference matrix \textit{I[p,q]} says if the paths started from \textit{p} and \textit{q} are going to intersect at some point. Authors implement their technique withing McCAT compiler, which uses SIMPLE intermediate representation with a total of 8 statements (\textit{malloc()}, pointer assignments \textit{p=q}, structure updates \textit{p-$>$next=q}), which are capable of changing \textit{D} and \textit{I} matrices. Statements generate and kill entries in matrices. Moreover, they are capable of changing \textit{Shape} attributes of pointers. The technique has been assessed on various benchmarks (bintree, xref, chomp, assembler, loader, sparse, etc.) from the era before the standard benchmark suites became available. The technique mostly reported shapes as \textit{Trees} (be it a binary tree or a linked-list) or sometimes as \textit{DAGs} or \textit{Cycles} but with higher error rates in these last cases. The latter shows that the technique is imprecise and conservative.\newline\null
\quad One of the more recent techniques designed and developed by Ginsbach et al. is based on the pattern matching on LLVM IR level. The main idea is to specify computational idioms to be recognized in a domain-specific constraint-based programming language CAnDL \cite{Ginsbach:2018:CDS:3178372.3179515}. Constraints are specified over LLVM IR entities such as instructions, basic blocks, functions, etc. The CAnDL language allows rapid prototyping of new compiler optimizations based on pattern recognition and its substitution with optimized versions of matched idioms. The language and its relatively fast backtracking constraint solver are capable of recognizing not only simple arithmetic idioms (thus performing different peephole optimizations), but more complex computations like general reductions and histograms \cite{Ginsbach:2017:DEG:3049832.3049862}, vector products in graphics shaders \cite{Ginsbach:2018:AML:3296957.3173182}, sparse and dense linear algebra computations and stencils \cite{Ginsbach:2018:AML:3296957.3173182}. Having recognized these computational idioms the work \cite{Ginsbach:2018:AML:3296957.3173182} replaces them with a code for various heterogeneous APIs (MKL, libSPMV, Halide, clBLAS, CLBlast, Lift) and compares the resulting performance demonstrating an improvement over sequential versions and matching performance to hand-written parallel versions. The technique has been deployed on the sequential C versions of SNU NPB, the C versions of Parboil, and the OpenMP C/C++ versions of Rodinia demonstrating improved detection capabilities over the state-of-the-art techniques.\newline\null
\quad The other principally different technique has been recently proposed by Changhee Jung and Nathan Clark \cite{1669122}. The authors developed a Data-structure Detection Tool (DDT) based on the LLVM framework. The tool instruments load, store, and call instructions within program binaries and gathers dynamic traces for sample inputs. The traces are used to recreate a memory allocation graph for program data structures. Call graphs are used to identify interface functions interacting with the built memory graph. DDT traces memory graph properties (number of nodes, edges, etc.) before and after interface function calls into another Daikon tool to compute dynamic invariants (the number of nodes in a memory graph decreases by 1 after every \textit(delete()) interface method call, etc.). In the end, manually constructed decision tree is used to probabilistically match observed behavioral patterns against known data structure invariant properties. The technique has been deployed to recognize data structure implementations within standard libraries like STL, Apache (STDCXX), Borland (STLport), GLib, Trimaran achieving almost perfect recognition accuracy. Moreover, the technique has been able to recognize linked lists in Em3d and Bh Olden benchmarks, along with red-black trees implementing vectors in the Xalancbmk benchmark.\newline\null
\quad There has recently been other published works on the application of dynamic techniques to the problem of dynamic data structure recognition \cite{Rupprecht:2017:DID:3155562.3155607}\cite{Haller:2016:SDS:2938006.2938029}. The technique used in the DDT tool \cite{1669122} makes an assumption, that all modifications and interactions with memory graphs representing data structures happen through a set of interface functions. That is not true when we deal with aggressively optimizing compilers, which may eliminate some code or inline some functions. The MemPick tool \cite{Haller:2016:SDS:2938006.2938029} searches data structures directly on a built dynamic memory graph by analyzing its shape. The graph is built with the help of the Intel Pin binary instrumentation tool during quiescent periods when pointer operations are absent. DSIbin tool \cite{Rupprecht:2017:DID:3155562.3155607} operates with the source code rather than program binaries. Instead of memory points-to graphs, it uses strands as primitives, which abstract such entities as singly-linked lists.\newline\null
\quad The work of Dekker \cite{Dekker:1994:ADS:3107859.3107876} addresses the software design recovery problem in a completely different way. Contrary to the approaches described above, which operate on the IR and dynamic instruction stream levels, the work of Dekker operates at the level of an abstract syntax tree. Dekker's tool tries to compact the tree down to recognizable syntactic patterns by transforming it under a special grammar.
\section{The discovery of algorithmic skeletons and parallel patterns}
\label{related_work_as_and_pp}
\quad The concept of algorithmic skeletons has been introduced in Section \ref{background_skeletons}. There is a vast array of various implementations and libraries. The latter vary by the programming language they target, distribution library used to implement parallel/distributed computations internally (like MPI, OpenMP, etc.), the sets of supported skeletons, and the allowed level and way of skeleton nesting, i.e composing more complex patterns of the basic ones; these include: RPL [27]; Feldspar [4]; FastFlow [2]; Microsoft’s Pattern Parallel Library [13]; and Intel’s Threading Building Blocks (TBB) library [35].
\newline\null

A range of pattern/skeleton implementations
have been developed for a number of programming languages; 


\quad Work \cite{skeletons-static} proposes a static approach and develops Parallel Pattern Analyzer Tool (PPAT) capable of analyzing sequential C++ code statically in order to detect and annotate parallel patterns. The tool takes advantage of a Clang library to generate an Abstract Syntax Tree (AST). Then, it walks through it finding AST loop candidates to be analyzed for possible parallel patterns. For every loop the tool checks a set of constraints specific for every target pattern. For example, for a loop in order to be a parallel pipeline, it must not write any global variables, pass no feedback between separate loop body stages, and there must be at least two stages to split the loop into. The tool uses a set of heuristics to determine the size of stages. For farm pattern, the tool checks that loop body has no RAW dependencies, no break statements, and writes no global variables. Authors evaluate the effectiveness and correctness of their approach using NAS and Rodinia benchmark suites by comparing automatically detected patterns against the manual analysis. Authors observed that the pattern detection quality of PPAT is close to that performed by a human expert. Then, the authors moved on manually parallelizing detected patterns demonstrating final performance comparable to expertly parallelized benchmark versions. The experimental evaluation demonstrates that PPAT is able to obtain similar performance results as the “handmade” parallel versions of the benchmark suites tested. Therefore, reducing the human effort in transforming sequential codes into parallel.

Some works take advantage of functional languages. For instance, István Bozó et al. \cite{10.1145/2633448.2633453} develop a tool that detects parallel patterns in applications written in Erlang. Compared to other languages, Erlang
features make the detection process much simpler. Nonetheless, the tool requires profiling techniques in order to decide which pattern suits best for a concrete problem.



The work \cite{10.1016/j.parco.2010.05.006} proposes
a tool that is based on profile-driven analysis and helps programmer to discover coarse-grained thread-level pipeline parallelism. The programmer hand-picks the code transformations from among the proposed candidates which are then applied by automatic code transformation techniques. The work is based on the belief that a significant amount of exploitable coarse-grain parallelism exists in outer loops of many programs with complex control and data flow. In contrast, with traditional auto-parallelization that focuses on parallelism within data structures (i.e. different threads operate on different elements of the same data structure), the tool addresses parallelizm spread across different data structures (i.e. different threads operate on different data structures).
The tool demonstrates the potential of pipeline parallelism for several MiBench and SPEC2000 integer benchmarks. Speedup
measurements on real hardware show that the techniques find scalable parallelism in many applications. Furthermore, the tool shows better results than those reported in the literature for both state-of-the-art speculative parallelization techniques \cite{10.1109/ISCA.2006.13} and more fine-grain parallelization techniques [].


Discovery of places in a sequential code, where parallel patterns might be introduces is highly non-trivial, often requiring a manual analysis and profiling. The work [] proposes a hybrid discovery technique to detect instances of parallel patterns in sequential code. The technique is based on dynamic traces together with hotspot detection. The work focuses on the two most well-known patterns of map and reduce.

\quad If we are coding 

\quad If one writes parallel software from scratch, a programming methodology based around parallel patterns can considerably alleviate the task [12, 48, 49].



The problem of legacy parallel code
coded parallel patterns ad hoc

does not fit modern heterogeneus architectures
low level non portable 


The work \cite{roberto-lozano-skeletons} aims to highlight to a programmer code fragments, which could be replaced by calls to known parallel pattern library abstractions of map, reduce and their compositions in a legacy Pthreaded C/C++ code. The underlying technique is based on the analysis of dynamic data flow graphs (DDFGs) obtained during execution of programs and thus is language-agnostic. The analysis uses constraint-based pattern matching to identify constrained DDFG subgraphs characteristic to the well-established parallel patterns while employing heuristics which trade analysis time against completeness making the approach scalable. The tool and methodology demonstrate an excellent effectiveness and accuracy on Starbench benchmarks by finding 36 out of the expected 42 instances of parallel patterns with a high accuracy (reporting actual patterns in 98\% of the cases). Authors re-express the found patterns via a parallel pattern library, making code freely portable across CPU/GPU systems and performing competitively with hand-tuned implementations at zero additional effort.\newline\null
\quad As the sequential code gives the cleanest starting point for the introduction of parallel patterns, the work \cite{} studies how parallel legacy C/C++ pthreaded codes could be converted back to sequential version and ultimately restored to a modern parallel pattern based equivalent form. This work studies the specifics of parallel legacy pthreaded codes and proposes a novel methodology to accomplish the task. The restoration is conducted through a systematic application of a number of identified program transformations. Authors design and define a set of restorative transformations common to many legacy pthreaded codes. The aim of these transformations is to (i) eliminate Pthread operations from legacy C/C++ programs; (ii) perform code repair, fixing any bugs introduced in (i); and, (iii) reshape code in preparation for parallel pattern introduction. 
The work targets only farm and pipeline patterns. The transformations presented in the work are intended as manual transformations. The implementation of these refactorings into a semi-automatic tool is envisaged as the future work. Authors use Intel TBB library to evaluate these transformations on a set of benchmarks and demonstrate that the removal of parallelism allows to manually derive structured parallel code that is comparable to the original legacy-parallel version in terms of performance, while being more portable, adaptive, and maintainable. Additionally authors record improvements in terms of cyclomatic complexity \cite{} and Lines Of Code (LOC) metric.

